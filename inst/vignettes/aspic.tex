% aspic.Rnw --
%
% Author: laurence kell <laurie@kell.es>

%\VignetteIndexEntry{aspic}
%\VignetteIndexEntry{An R Package for run ASPIC and reading/wrting ASPIC files}
%\VignetteKeyword{aspic, read, write}

\documentclass[shortnames,nojss,article]{jss}

\usepackage[utf8]{inputenc}
%\usepackage{hyperref}
%\usepackage{geometry}
\usepackage{framed}
\usepackage{color}
\usepackage[onehalfspacing]{setspace}
\usepackage{natbib} \bibliographystyle{plain}
\usepackage{graphicx, psfrag, Sweave}
%\usepackage{enumerate}
%\geometry{verbose,a4paper,tmargin=2cm,bmargin=1.5cm,lmargin=2cm,rmargin=3cm}

%\usepackage{booktabs,flafter} %,thumbpdf}

\definecolor{shadecolor}{rgb}{0.9,0.9,0.9}

\newenvironment{mylisting}
{\begin{list}{}{\setlength{\leftmargin}{1em}}\item\scriptsize\bfseries}
{\end{list}}

\newenvironment{mytinylisting}
{\begin{list}{}{\setlength{\leftmargin}{1em}}\item\tiny\bfseries}
{\end{list}}

\author{Laurence Kell\\ICCAT }
\Plainauthor{Laurence Kell}

\title{\pkg{aspic}: Biomass Dynamic Stock Assessment Model}
\Plaintitle{aspic: Biomass Dynamic Stock Assessment Model}

\Abstract{The \pkg{aspic} package is an implenentation of the ASPIC biomass dynamic stock assessment model in 
R using the original \proglang{FORTRAN} executable. The package provides tools for checking of diagnostics, projections, 
running Monte Carlo simulation and conducting Management Strategy Evaluation.}

\Keywords{aspic, \proglang{R},  stock assessment}
\Plainkeywords{aspic, R stock assessment}

%% need no \usepackage{Sweave.sty}

%\newcommand{\code}[1]{\texttt{#1}}
%\newcommand{\proglang}[1]{\textsf{#1}}
%\newcommand{\pkg}[1]{{\fontseries{b}\selectfont #1}}

%% need no \usepackage{Sweave.sty}

\Address{
  Laurence Kell \\
  ICCAT Secretariat\\ 
  C/Coraz\'{o}n de Mar\'{\i}a, 8. \\
  28002 Madrid\\
  Spain\\ 
  
  E-mail: \email{Laurie.Kell@iccat.int}
}

%% need no \usepackage{Sweave.sty}



\begin{document}
\Sconcordance{concordance:aspic.tex:aspic.Rnw:%
1 66 1 1 20 32 1 1 2 1 0 1 4 2 0 2 1 1 2 2 1 1 3 5 0 1 3 15 1 1 2 1 0 1 %
4 2 0 2 1 1 2 2 1 1 3 5 0 1 3 3 1 1 4 3 0 2 1 1 2 2 1 1 3 1 0 2 1 3 0 1 %
2 5 1 1 3 1 0 2 1 1 6 4 0 1 2 1 5 6 0 1 2 2 1 1 2 1 0 1 4 6 0 1 2 3 1 1 %
2 1 0 6 1 1 2 1 1 1 6 8 0 1 2 2 1 1 2 1 0 2 1 1 2 2 1 1 4 6 0 1 2 2 1 1 %
2 1 0 2 1 1 2 4 0 1 2 5 1 1 2 4 0 1 2 3 1 1 2 4 0 1 2 2 1 1 3 4 1 1 3 3 %
1 1 3 4 1 1 3 2 1 1 3 19 1 1 3 7 1 1 3 4 1 1 3 4 1 1 3 6 1 1 3 52 1}


\newpage\tableofcontents\newpage

\section{Introduction}

\pkg{aspic} is a biomass dynamic model \citep{prager_suite_1994,prager_aspic-surplus-production_1992} originially implemented as a \proglang{FORTRAN} executable.
The package provides tools for checking of diagnostics, projections, running Monte Carlo simulation and conducting Management Strategy Evaluation.
Since the original \proglang{FORTRAN} code uses proprietry source code it can not be released under an open source licence and so only the
\proglang{FORTRAN} execuatable can be provided as part of the package. In consequence the package is not avialable via \href{}{CRAN} but
from \href{}{CRAN} instead.

Making \pkg{aspic} available as a pakage allows it to be used with other packages for plotting, summarising results and to be simulation tested, e.g. 
as part of the FLR tools for management strategy evaluation \cite{kell_evaluation_1999}; see 
\citep{kell2013mpalbn,kell2013diagalbn,kell2013diagalbs,kell2013msealbn,kell2013msealbn,kell2013fwdalbn,kell2013diagswon,kell2013diagswos,kell2013fwdswon,kell2013profile,kell2013uncert}

\pkg{aspic} depends upon the \cite{biodyn} package, which contains most of the methods for  
\section{Data}

\pkg{aspic} uses data on catch and catch per unit effort (CPUE) to estimate model parameters using the simplex algorithm to minimise the residual 
sum of squares from the fit observed CPUE. The parameters estimated are catchability (q) by fleet and the parameters of the suplus production 
function. 

Data can either be provided as a data.frame or as a test input file. 

\subsection{R}

The best way is to provide input data as a data.frame, with columms for \code{year,catch,cpue} and \code{code}; where the 
later column indicates the type of index (\textbf{Table} \ref{code}).

\begin{mylisting}\begin{center}\begin{minipage}[H]{0.95\textwidth}\begin{shaded} 
\begin{Schunk}
\begin{Sinput}
> library(aspic)
> ### Assessments
> ## 1 file
> aspic=readASPIC(paste(dirAspic,"/",scen=scen[1],".bio",sep=""))
> class(aspic)
> names(aspic)
> aspic=readASPIC(paste(dirAspic,"/",scen=scen[1],".bio",sep=""),data.frame=T)
> class(aspic)
> names(aspic)
> ## many files
> aspics=readASPIC(dirAspic,scen=scen,type="b",data.frame=T)
> 
\end{Sinput}
\end{Schunk}
\end{shaded}\end{minipage}\end{center}\end{mylisting}



\subsection{Files}

There are six types of files, i.e.\\

 \textbf{.bio} bootstrap estimates of historic biomass and harvest rate\\
 \textbf{.prj}  bootstrapped projections with predicted biomass and harvest rates\\
 \textbf{.det}  parameter estimates by bootstrap trial\\
 \textbf{.inp}  the input file with data, starting guesses, and run settings and for output\\
 \textbf{.prb}  as .bio but with projection results\\


\begin{mylisting}\begin{center}\begin{minipage}[H]{0.95\textwidth}\begin{shaded} 
\begin{Schunk}
\begin{Sinput}
> library(FLAdvice)
> ### Assessments
> ## 1 file
> aspic=readASPIC(paste(dirAspic,"/",scen=scen[1],".bio",sep=""))
> class(aspic)
> names(aspic)
> aspic=readASPIC(paste(dirAspic,"/",scen=scen[1],".bio",sep=""),data.frame=T)
> class(aspic)
> names(aspic)
> ## many files
> aspics=readASPIC(dirAspic,scen=scen,type="b",data.frame=T)
> 
\end{Sinput}
\end{Schunk}
\end{shaded}\end{minipage}\end{center}\end{mylisting}


\begin{mylisting}\begin{center}\begin{minipage}[H]{0.95\textwidth}\begin{shaded} 
\begin{Schunk}
\begin{Sinput}
> #### Projections
> ## 1 file
> prj=readASPIC(paste(dirAspic,"/","bumcont1bproj500",".prj",sep=""))
> class(prj)
> names(prj)
> prj=readASPIC(paste(dirAspic,"/","bumcont1bproj500",".prj",sep="",data.frame))
> class(prj)
> names(prj)
> ## many
> prjs=readASPIC(dirAspic,scen=expand.grid(scen=c("bumcont1bproj","bumhighpproj"),TAC=seq(0,6000,500)))
> class(prjs)
> names(prjs)
\end{Sinput}
\end{Schunk}
\end{shaded}\end{minipage}\end{center}\end{mylisting}

\subsection{R}

There is an example text data set 
\begin{mylisting}\begin{center}\begin{minipage}[H]{0.95\textwidth}\begin{shaded} 
\begin{Schunk}
\begin{Sinput}
> dirInp=paste(system.file(package="aspic"),"extdata",sep="/")
> asp=aspic(paste(dirInp,"albn.inp",sep="/"))
> asp=fit(asp)
> key=data.frame(
+   name   =c("Troll Composite CPUE","JLL Old","JLL Modern","CT Old","CT Modern"), 
+   series=c("I","I","II","I","II"),
+   flag  =c("OT","JA","JA","CT","CT"),
+   gear  =c("TR","LL","LL","LL","LL"))
> dimnames(key)[[1]]=c("Troll Composite CPUE","JLL Old","JLL Modern","CT Old","CT Modern") 
> wts=t(array(c(1,1,1,1,1,
+               1,0,0,0,0,
+               0,1,1,0,0,
+               0,0,0,1,1),c(5,4),list(name=key$name,Scenario=1:4)))
\end{Sinput}
\end{Schunk}
\end{shaded}\end{minipage}\end{center}\end{mylisting}

\begin{mylisting}\begin{center}\begin{minipage}[H]{0.95\textwidth}\begin{shaded} 
\begin{Schunk}
\begin{Sinput}
> cpue=subset(diags(asp),!is.na(obs))[,c("year","name","obs")]
> ggplot(aes(year,obs,group=name,col=name),data=cpue)+
+   geom_point()+
+   stat_smooth()+
+   theme_ms(legend.position="bottom")
\end{Sinput}
\end{Schunk}
\end{shaded}\end{minipage}\end{center}\end{mylisting}


\begin{mylisting}\begin{center}\begin{minipage}[H]{0.95\textwidth}\begin{shaded} 
\begin{Schunk}
\begin{Sinput}
> library(gam)
> gm  =gam(log(obs)~lo(year)+name,data=cpue)
> cpue=data.frame(cpue,gam=predict(gm),gamRsdl=residuals(gm))
> scl =coefficients(gm)[3:9]
> names(scl)=substr(names(scl),5,nchar(names(scl)))
> cpue=transform(cpue,scl=scl[as.character(name)])
> cpue[is.na(cpue$scl),"scl"]=0
> cpue=cbind(cpue,key[cpue$name,])[,-2]
> cpue$name=factor(cpue$name, levels=c("Troll Composite CPUE","JLL Old","JLL Modern","CT Old","CT Modern"))
> ggplot(cpue)+ geom_line(aes(year,exp(gam)),col="red")  +
+               geom_smooth(aes(year,obs),se=FALSE)      +           
+               geom_point( aes(year,obs,col=name))       +
+               facet_wrap(~name,ncol=1,scale="free_y")  +
+               theme_ms(legend.position="none")         +
+               xlab("Year") + ylab("Index")
\end{Sinput}
\end{Schunk}
\end{shaded}\end{minipage}\end{center}\end{mylisting}

\begin{mylisting}\begin{center}\begin{minipage}[H]{0.95\textwidth}\begin{shaded} 
\begin{Schunk}
\begin{Sinput}
> uMat=ddply(cpue,.(name),transform, obs=stdz(obs))
> uMat=cast(uMat,year~name,value="obs")
> uMat=uMat[apply(uMat,1,function(x) !all(is.na(x))),]
> pM=plotmatrix(uMat[,-1])
> pM$layers[[2]]=NULL
> mns=ddply(subset(pM$data,!(is.na(x) & !is.na(y))),.(xvar,yvar), function(x) mean(x$y,na.rm=T))
> pM+geom_hline(aes(yintercept=V1),data=mns,col="red") +
+    geom_smooth(method="lm",se=F)  + 
+    theme(legend.position="bottom")                   +
+    xlab("Index")+ylab("Index")      
\end{Sinput}
\end{Schunk}
\end{shaded}\end{minipage}\end{center}\end{mylisting}

\begin{mylisting}\begin{center}\begin{minipage}[H]{0.95\textwidth}\begin{shaded} 
\begin{Schunk}
\begin{Sinput}
> cr=cor(uMat[,-1],use="pairwise.complete.obs")
> dimnames(cr)=list(gsub("_"," ",names(uMat)[-1]),gsub("_"," ",names(uMat)[-1]))
> cr[is.na(cr)]=0
> corrplot(cr,diag=F,order="hclust",addrect=2)  +          
+              theme(legend.position="bottom")  
\end{Sinput}
\end{Schunk}
\end{shaded}\end{minipage}\end{center}\end{mylisting}

\section{Assessment}


\begin{mylisting}\begin{center}\begin{minipage}[H]{0.95\textwidth}\begin{shaded} 
\begin{Schunk}
\begin{Sinput}
> asp=fit(asp)
\end{Sinput}
\end{Schunk}
\end{shaded}\end{minipage}\end{center}\end{mylisting}


\begin{mylisting}\begin{center}\begin{minipage}[H]{0.95\textwidth}\begin{shaded} 
\begin{Schunk}
\begin{Sinput}
> plot(asp)
\end{Sinput}
\end{Schunk}
\end{shaded}\end{minipage}\end{center}\end{mylisting}

\begin{mylisting}\begin{center}\begin{minipage}[H]{0.95\textwidth}\begin{shaded} 
\end{shaded}\end{minipage}\end{center}\end{mylisting}

\section{Diagnostics}

\begin{mylisting}\begin{center}\begin{minipage}[H]{0.95\textwidth}\begin{shaded} 
\end{shaded}\end{minipage}\end{center}\end{mylisting}


\begin{mylisting}\begin{center}\begin{minipage}[H]{0.95\textwidth}\begin{shaded} 
\end{shaded}\end{minipage}\end{center}\end{mylisting}

\section{Reference Points}

\begin{mylisting}\begin{center}\begin{minipage}[H]{0.95\textwidth}\begin{shaded} 
\end{shaded}\end{minipage}\end{center}\end{mylisting}

\begin{mylisting}\begin{center}\begin{minipage}[H]{0.95\textwidth}\begin{shaded} 
\end{shaded}\end{minipage}\end{center}\end{mylisting}

\section{Fitting}

\section{Plotting}

There are various standard plots, i.e. for fitted time series, reference points and diagnostics. Also using ggplot2 a variety of ad-hoc plots can be produced as required and the packages \pkg{diags} and \pkg{kobe} can be used for diagnostics and providing plots in Kobe II advice framework. 

\subsection{CPUE}

\subsection{Diagnostics}

\subsubsection{Residuals}
\subsubsection{Likelihood Profiling}

\section{Uncertainty}

\subsection{Bootstrapping}

\begin{mylisting}\begin{center}\begin{minipage}[H]{0.95\textwidth}\begin{shaded} 
\end{shaded}\end{minipage}\end{center}\end{mylisting}


\section{Management Procedure}

\subsection{Reference points}

\begin{mylisting}\begin{center}\begin{minipage}[H]{0.95\textwidth}\begin{shaded} 
\end{shaded}\end{minipage}\end{center}\end{mylisting}

\subsection{Projections}

\begin{mylisting}\begin{center}\begin{minipage}[H]{0.95\textwidth}\begin{shaded} 
\end{shaded}\end{minipage}\end{center}\end{mylisting}

\subsection{Harvest Control Rules}

\begin{mylisting}\begin{center}\begin{minipage}[H]{0.95\textwidth}\begin{shaded} 
\end{shaded}\end{minipage}\end{center}\end{mylisting}

\section{Advice}

\subsection{Kobe Framework}

\begin{mylisting}\begin{center}\begin{minipage}[H]{0.95\textwidth}\begin{shaded} 
\end{shaded}\end{minipage}\end{center}\end{mylisting}

\section{MSE}

\bibliography{/home/laurie/Desktop/scrs2014/include/tex/refs.bib} 
\bibliographystyle{abbrvnat} 

\begin{table}
\caption{Equations}
\label{tab:datasumm}
\begin{tabular}{lp{10cm}}
\toprule
\toprule

\textbf{Population dynamics} &
 \begin{equation} N_{a+1, y+1} = N_{a,y} e^{-Z_{a,y}} \end{equation} \\
%
 &
 \begin{equation} N_{p,y} = N_{p-1, y-1} e^{-Z_{p-1, y-1}} + N_{p,y} e ^{-Z_{p, y-1}} \end{equation} \\
%
 &  \begin{equation} N_{r,y} = f(B_{y-r}) \end{equation} \\
\midrule

% Mortality rates
\textbf{Mortality rates} & \begin{equation} Z_{a,y} = F_{a,y} + D_{a,y} + M_{a,y} \end{equation} \\
%
 & \begin{subequations} 
\begin{equation} F_{a,y} = \sum_{i=1}^f P_{i,a,y} S_{i,a,y} E_{i,y} \end{equation}
\begin{equation} D_{a,y} = \sum_{i=1}^f \left(1- P_{i,a,y}\right) S_{i,a,y} E_{i,y} \end{equation}
\end{subequations}\\
\midrule

% Catch equation
\textbf{Catch equation} & \begin{equation} C_{f,a,y} = N_{a,y} \frac{F_{f,a,y}}{Z_{f,a,y}} \left(1 - e^{-Z_{a,y}} \right) \end{equation} \\
\midrule

%
\multicolumn{2}{l}{\textbf{Stock recruitment relationships}} \\
\addlinespace
% BH
Beverton \& Holt & \begin{equation} N_{r,y} = \frac{B_{y-r}}{\alpha B_{y-r} + \beta} \end{equation} \\
% BB
\bottomrule

\multicolumn{2}{l}{\textbf{Growth and maturity}} \\
% BH
von Bertalanffy & \begin{equation} N_{r,y} = \frac{B_{y-r}}{\alpha B_{y-r} + \beta} \end{equation} \\
\bottomrule
\end{tabular}
\end{table}

\end{document}

